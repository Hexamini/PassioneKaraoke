\subsection{HTML}
La struttura del sito \`e stata realizzata a ``moduli'': non esistono infatti pagine complete, ma son presenti spezzoni di pagine che servono al Perl per generare la pagina richiesta dall'utente. Questo permette un riutilizzo dei vari moduli, come ad esempio la lista di tutti gli album di un artista.
\subsubsection{Strutturazione}

%da scrivere: bisogna vedere la pagina web come una struttura ad albero, dove la radice e' page e tutto il resto son nodi figli

\paragraph*{MetaTag}Le pagine web presentano una gerarchia tra di loro, con il file page.html come nodo padre (radice). Per generare quindi correttamente i meta-tag si \`e deciso di inserire le informazioni nei nodi figli (sempre tramite segnaposti). Questo permette allo script \textit{r.cgi}, durante la costruzione della pagina web, di prendere le informazioni contenute nei nodi figli e di sostituirli al valore del segnaposto del padre.

%IDEA: esprimere le funzionalità delle pagine principali del sito
\paragraph*{page.html} Il file page.html \`e il file principale per tutta la struttura html. In esso infatti sono presenti la dichiarazioni di html usato (per la realizzazione di questo sito si \`e deciso di utilizzare XHTML 1.0 Strict), i tag \textit{meta}, il \textit{breadcrum}, il \textit{menu}. Tutte le altre pagine vengono costruite a partire da page.html: questo fa variare solamente parte della struttura, permettendo di apportare modifiche al men\`u o alle altre componenti condivise da tutte le pagine del sito in maniera semplice.

\paragraph*{userPage.html} Il file userPage.html si occupa della gestione della sezione utente, dove potrà vedere i voti apportati alle singole canzoni. L'utente ha la possibilità di apportare un voto positivo (+1) o negativo (-1) cliccando rispettivamente il microfono blu o rosso nella pagina della canzone.

\paragraph*{songDescription.html} Questa struttura html si occupa di rappresentare le canzoni (i contenuti saranno effettivamente forniti dal Perl tramite l'uso dei \textit{segnaposti}). Subito al di sotto del breadcrum è possibile trovare il titolo della canzone con il rispettivo artista come sottotitolo. Per gli utenti autenticati sarà possibile trovare al di sotto del titolo l'opzione per esprimere il proprio gradimento. Il contenuto, ovvero le \textit{lyrics} si trovano subito sotto, seguite infine da un video\footnote{È importante notare come i video saranno resi fruibili tramite la piattaforma \textit{YouTube}} che può essere inserito o meno a discrezione dell'amministratore. I video saranno visualizzabile solamente dai browser che supportano questa funzionalità, altrimenti non verranno visualizzati, causando una buona degradazione del sito.

\paragraph*{artist.html} artist.html è la struttura HTML che permette di rappresentare il profilo di un singolo artista. Il cantautore, o la band, sarà accompagnata, dopo un breve descrizione, dalla lista degli album scritti presente nel sito. Non è possibile da parte dell'utente esprimere qualsiasi tipo di preferenze nè applicare qualsiasi modifica, che invece sarà possibile dall'amministratore tramite la \textit{modalità amministrativa}.

\paragraph*{album.html} In questa struttura HTML viene rappresentato un album specifico. All'interno della rappresentazione di un album si trovano le liste di tutte le canzoni appartenenti a esso. È possibile accedere per gli utenti autorizzati alla modalità ``amministrativa'', dando la possibilità di modificare i dati relativi all'album e la lista delle canzoni appartenenti a esso. %paragrafo da rivedere

\paragraph*{404.html} La pagina 404 viene visualizzata nel sito web quando si capita in una risorsa non presente nel server. La pagina 404 è una struttura statica, che riporta semplicemente una frase di errore cercando di divertire l'utente sull'accaduto.

%TODO
\subsubsection{Scelte effettuate nella strutturazione}
%TODO

\paragraph{HTML5}
È stato deciso di utilizzare HTML5 in quanto ha permesso l'utilizzo dei frame video. Oltre questa funzionalità, è stato utilizzato codice HTML quanto più aderente all'XHTML 1.0 Strict, assicurando quindi ai visitatori aventi browser datati un'ottima degradazione del sito e senza perdita di funzionalità tranne che per i video nella pagina \textit{songDescription.html}.
%TODO: che altro dire sull'html5?
