\subsection{Javascript}
In fase di progettazione si \`e deciso di limitare il pi\`u possibile l'utilizzo della tecnologia Javascript, che \`e stata sfruttata esclusivamente per i controlli sui campi form, nei quali garantisce una maggiore usabilit\`a a livello utente. Questo permette infatti di segnalare eventuali errori nella compilazione dei campi direttamente durante l'inserimento dei dati senza necessit\`a di ricaricare la pagina che sarebbe visto negativamente dall'utente, se non altro per la perdita di tempo che questo comporta. La scelta di limitare questa tecnologia \`e stata guidata dalla necessit\`a di fornire al sito caratteristiche che lo rendessero quanto pi\`u possibile accessibile al grande bacino di utenti che lo visiteranno. Evitare l'utilizzo di Javascript nel resto del sito, infatti, consente di guadagnare accessibilit\`a per quegli utenti che utilizzano screen reader. Com'/`e noto infatti, la maggior parte di queste applicazioni software non riescono a eseguire Javasript e quindi gli avvisi stampati con questa tecnologia non sarebbero visualizzati dall'utente, il tutto ovviamente con una grossa perdita di informatori per quei visitatori che presentano deficit visivi.
Da sottolineare comunque che tutte le funzionalit\`a non presenti quando la tecnologia JavaScript \`e disattivata sono state fornite tramite il Perl. \'E una scelta che permette di non perdere funzionalit\`a a scapito per\`o del fatto che, utilizzando Perl, la pagina dev'essere ricaricata perch\'e gli errori vengano segnalati e la gestioni degli errori non /`e pi/`u a livello locale, della macchina, ma viene gestita dal server.
