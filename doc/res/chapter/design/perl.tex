\subsection{Perl}
Il perl \`e il motore del sito, che ne permette il funzionamento offrendo pagine html all'utente, e gestisce le query sulla base XML garantendo la compatibilit\`a tra dispositivi in quando eseguito lato server.
\subsubsection{Generazione delle pagine web}
\textit{The biggest cause of unmaintainable web sites is mixing markup code such as HTML and program code such as Perl.}\footnote{[cit. \textit{Perl Template Toolkit}]}\\

La netta divisione tra le tre componenti di un sito web (\textit{Presentazione}, \textit{Struttura}, \textit{Comportamento}) \`e un traguardo ambito ma non semplice da raggiungere. Se il CSS rende semplice la divisione tra struttura e presentazione, lo stesso non si pu\`o dire del Perl: contrariamente al CSS, dove si ha una netta separazione tra struttura e presentazione, \`e necessario prestare attenzione all'interazione con l'HTML. Si \`e quindi adottato uno sviluppo tramite modularizzazione, che ha permesso di raggiungere un buon traguardo.

\paragraph*{Descrizione della soluzione adottata}Una non chiara seprarazione tra struttura e comportamento sta nella mancanza di un supporto nativo per Perl ed HTML, che in grado di dare la possibilit\`a in modo \textit{semplice e bello} di definire una procedura per il popolamento di una pagina Web. L'obiettivo degli script creati durante la relizzazione del progetto \`e riempire spazi vuoti in aree \textit{statiche} costrette ad essere considerate come \textit{dinamiche} per una limitazione espressiva del linguaggio. Da qui \`e nata l'esigenza di ampliare il linguaggio HTML in modo di dare la possibilit\`a al Perl di riferirsi ad una determinata area dove inserire il contenuto d'interesse. La soluzione \`e stata raggiunta adottando l'utilizzo della libreria \textit{Perl Template Toolkit}, che rende possibile in maniera agevole l'inseriemento di \textit{segnaposti} all'interno della struttura HTML. Gli script, infine, si occuperanno di sostituire i \textit{segnaposti} con l'effettivo contenuto.
\paragraph*{Segnaposti} Qui un esempio di segnapostom, racchiuso tra la sequenza '[\%' e '\%]', applicato alla struttura HTML. 
\lstinputlisting[language=HTML,caption=Esempio di segnaposto]{res/codeExample/segnaposto.html} %c'e' gia' da un'altra parte della relazione. Meglio un richiamo?

Prima di visualizzare l'HTML \`e necessario applicare il parsing del Perl Template Toolkit in moda da sostituire i simboli di segnaposto con il valore richiesto sia esso solo testo o struttura HTML presi da altri file.
La Perl Template Toolkit dispone una serie di utilities a livello di sintassi dette ``\textit{Direttive}'', per offrire una serie di costrutti (es. ciclo for, while, ecc.) in grado di renderlo equiparabile alle trasformate XSLT: durante la stesura del progetto si \`e deciso di non implementare questa soluzione in quanto si si \`e temuto di mescolare il comportamento con la struttura.


\subsubsection{Organizzazione}
Durante lo sviluppo del codice Perl si \`e puntato alla ricerca del riuso e dell'estendibilit\`a. Ispirandosi al paradigma della programmazione orientata agli oggetti si \`e applicato il pattern MVC, per poter modellare la ``gerarchia'' di elementi web in maniera efficacie ed efficente.

\paragraph*{Funzionamento del pattern MVC}Ogni elemento definito presenta tre componenti:
\begin{enumerate}
    \item viev HTML (\textit{Model});
    \item regole CSS sui tag definiti (\textit{View});
    \item package Perl contenente le regole di costruzione e funzionalit\`a richieste (\textit{Controller}).
\end{enumerate}

Data la mancanza delle classi nel linguaggio Perl, in sostituzione, sono stati utilizzati i \textit{Package}. Alla istanziazione di un package viene costruita la pagina web che viene poi ottenuta tramite il metodo \textit{get}, unico metodo che funge da costruttore.

La OOP\footnote{Acronimo di \textit{Programmazione Orientata agli Oggetti}.} viene applicata nella definizione della gerarchia citata. Ogni package ``estende'' il package \textit{Base}, dove sono archiviati tutti gli script base per il funzionamento del sito web e metodi di uso comune. Di fatto nel package Base \`e possibile trovare:
\begin{itemize}
\item script di parsing della view HTML usando Perl Template Toolkit;
\item script per la lettura dei file xml;
\item script per la gestione delle sessioni;
\item script per il concatenamento tra i vari oggetti;
\end{itemize}

Da Base si deriva in \textit{Object} che, sfuttando le funzionalit\`a della classe padre, costruisce tutti gli elementi presenti nel sito web. L'oggetto \textit{Page} si occupa di mettere in relazione gli oggetti tra di loro per formare le pagine da visualizzare. Qui si pu\`o notare la reale versatilit\`a nell'utilizzo segnaposti: \`e possibile definire la pagina come l'insieme di segnaposti e con Perl caricare ogni modulo \underline{senza legare} comportamento con struttura.


%Durante lo sviluppo \`e capitato di modificare parti di HTML o molto pi\`u spesso quelle di Perl ma in entrambi i casi le operazioni non erano dispendiose in quanto un cambio nella logica non implicava un cambio nel modello, e viceversa. %sta frase mi sembra inutile. Mi pare ovvio che in un progetto avvengano delle modifiche :D


%TODO: da rileggere e controllare

\subsubsection{Visualizzazione e navigazione}

L'html \`e stato scritto in maniera tale da permettere una massima modularizzazione. \textbf{r.cgi} \`e lo script principale del sito, che si occupa di servire le pagine corrette.
Le pagine web che vengono servite sono ``costruite" dal perl secondo uno specifico metodo. \textit{r.cgi} chiama il modulo per la costruzione della pagina, che ricorsivamente chiama il modulo per la costruzione delle sottoparti, restituendo poi il tutto all'utente.

Durante il parsing, viene riconosciuto il segnaposto e viene aggiunto il modulo html corrente.
\subsubsection{Interazione con la base XML}
Perl non si occupa solamente della generazione dell'html, ma anche del prelievo dei dati dalla base scritta in XML.
%TODO
\subsubsection{Gesione delle sessioni}
Le sessioni (salvate nel server) vengono gestite tramite l'utilizzo di cookies (che riportano l'id della sessione corrispondente): permettono agli utenti registrati di eseguire il login e di accedere ad altre funzionalit\`a.
%TODO
