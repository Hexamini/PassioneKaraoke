\subsection{Perl}
Il perl \`e il motore principale del sito, che ne permette il funzionamento offrendo pagine html all'utente, e gestisce le query sulla base XML.
\subsubsection{Generazione delle pagine web}
L'html \`e stato scritto in maniera tale da permettere una massima modularizzazione. \textbf{r.cgi} \`e lo script principale del sito, che si occupa di servire le pagine corrette.
Le pagine web che vengono servite sono ``costruite" dal perl secondo uno specifico metodo. \textit{r.cgi} chiama il modulo per la costruzione della pagina, che ricorsivamente chiama il modulo per la costruzione delle sottoparti, restituendo poi il tutto all'utente.
\paragraph*{Segnaposti} Per poter realizzare ci\`o sono stati implementati l'utilizzo di segnaposti, che permettono tramite un parsing di includere le varie parti dell'html correttamente.
\lstinputlisting[language=HTML,caption=Esempio di segnaposto]{res/codeExample/segnaposto.html}
Durante il parsing, viene riconosciuto il segnaposto e viene aggiunto il modulo html corrente.
\subsubsection{Interazione con la base XML}
Perl non si occupa solamente della generazione dell'html, ma anche del prelievo dei dati dalla base scritta in XML.
%TODO
\subsubsection{Gesione delle sessioni}
Le sessioni (salvate nel server) vengono gestite tramite l'utilizzo di cookies (che riportano l'id della sessione corrispondente): permettono agli utenti registrati di eseguire il login e di accedere ad altre funzionalit\`a.
%TODO
