\subsection{Perl}
Il perl \`e il motore del sito, che ne permette il funzionamento offrendo pagine html all'utente, e gestisce le query sulla base XML garantendo la compatibilit\`a tra dispositivi in quando eseguito lato server.
\subsubsection{Generazione delle pagine web}
,,\textit{The biggest cause of unmaintainable web sites is mixing markup code such as HTML and program code such as Perl.}``\\

La netta divisione tra le tre componenti di un sito web ( \textit{Presentazione}, \textit{Struttura}, \textit{Comportamento} ) \`e un traguardo ambito ma non semplice da raggiungere. Se il CSS rende semplice la divisione tra struttura e presentazione, stesso non si pu\`o dire del Perl. L'uso dei riferiemti ad attributi id o class prensenti nei tag HTML permette un'integrazione trasparente con il CSS mentre per legare Perl con HTML \`e necessario dover produrre la struttura direttamente negli script preponenti. La mescolazione tra struttura e comportamento ne \`e una conseguenza diretta e seppur con qualche buona pratica di programmazzione ( es. modularizzazione ) sia possibile minimizzare i rischi, non si raggiunge comunque lo scopo di scindere completamente questi due macromondi.

L'origine di ci\`o sta nella mancanza di un supporto nativo per Perl ed HTML in grado di dare la possibilit\`a in modo \textit{semplice e bello [cit.]} di definire una procedura per il popolamento di una pagina Web. L'uso degli script nel progetto \`e riempire spazi vuoti in aree \textit{statiche} costrette ad essere considerate come \textit{dinamiche} per una limitazione espressiva del linguaggio.

\`E dai problemi sopra citati che \`e nata l'esigenza di ampliare il linguaggio HTML in modo di dare la possibilit\`a al Perl di riferirsi ad una determinata area dove inserire il contenuto d'interesse. La risposta a questa richiesta arriva dalla libreria \textit{Perl Template Toolkit}, che semplicemente rende possibile l'inseriemento di \textit{segnaposti} all'interno della struttura HTML, a cui basta in Perl riferirsi ( similmente a come fa CSS o JavaScript con id e class ) per applicarci il valore desiderato.
\paragraph*{Segnaposti} Qui un esempio di segnapostom, racchiuso tra la sequenza '[\%' e '\%]', applicato alla struttura HTML. 
\lstinputlisting[language=HTML,caption=Esempio di segnaposto]{res/codeExample/segnaposto.html}

Prima di visualizzare l'HTML \`e necessario applicare il parsing del Perl Template Toolkit in moda da sostituire i simboli di segnaposto con il valore richiesto sia esso solo testo o struttura HTML presi da altri file.
La Perl Template Toolkit offre una serie di utilities a livello di sintassi definite \textit{Direttive} per offrire una serie di costrutti ( es. ciclo for, while, ecc. ) che lo rendono efficacie quanto le trasformate XSLT. Tuttavia in questo progetto non vengono applicate in quanto pensiamo che porterebbe l'effetto opposto a quello desiderato, ovvero, mescolare il comportamento con la struttura.

\subsubsection{Organizzazione}
Un'aspetto importante che si \`e tenuto conto nel progetto \`e stata il fattore di estendibilit\`a e di riuso. Ci siamo ispirati al paradigma di programmazzione orientato agli oggetti per poter modellare la nostra 'gerarchia' di elementi web ed usato il pattern architetturale MVC in quanto si sposa magnificamente con le componenti di un sito web.

Ogni elemento definito avr\`a tre componenti: la view HTML ( Model ), regole CSS sui tag definiti ( View ) ed il package Perl contenente le regole di costruzione e funzionalit\`a richieste ( Controller ).
I package sono stati introdotti in sostituzione alle classi visto che l'unico momento di un oggetto web \`e l'istanziazzione per la visualizzazione, infatti in ogni package \`e presente il metodo \textit{get} che funge da costruttore.

La OOP viene applicata nella definizione della gerarchia citata. Ogni package si pu\`o dire che derivi dal package \textit{Base} dove sono archiviati tutti gli script base per il funzionamento del sito web e metodi di uso comune. Di fatto in base troviamo:
\begin{itemize}
\item Script di parsing della view HTML usando Perl Template Toolkit;
\item Script per la lettura dei file xml;
\item Script per la gestione delle sessioni;
\item Script per il concatenamento tra i vari oggetti;
\end{itemize}

Da Base si deriva in \textit{Object} che sfuttando le funzionalit\`a della classe padre si costruiscono tutti gli elementi presenti nel sito web. Infine c\`e \textit{Page} che mette in relazione gli oggetti tra di loro per formare le pagine da visualizzare. \`E qui che si dimostra la potenza dei segnaposti in quando \`e possibile definire la pagina come l'insieme di segnaposti e con Perl caricare ogni modulo senza legare comportamente con struttura.
Durante lo sviluppo \`e capitato di modificare parti di HTML o molto pi\`u spesso quelle di Perl ma in entrambi i casi le operazioni non erano dispendiose in quanto un cambio nella logica non implicava un cambio nel modello, e viceversa.

\subsubsection{Visualizzazione e navigazione}




L'html \`e stato scritto in maniera tale da permettere una massima modularizzazione. \textbf{r.cgi} \`e lo script principale del sito, che si occupa di servire le pagine corrette.
Le pagine web che vengono servite sono ``costruite" dal perl secondo uno specifico metodo. \textit{r.cgi} chiama il modulo per la costruzione della pagina, che ricorsivamente chiama il modulo per la costruzione delle sottoparti, restituendo poi il tutto all'utente.

Durante il parsing, viene riconosciuto il segnaposto e viene aggiunto il modulo html corrente.
\subsubsection{Interazione con la base XML}
Perl non si occupa solamente della generazione dell'html, ma anche del prelievo dei dati dalla base scritta in XML.
%TODO
\subsubsection{Gesione delle sessioni}
Le sessioni (salvate nel server) vengono gestite tramite l'utilizzo di cookies (che riportano l'id della sessione corrispondente): permettono agli utenti registrati di eseguire il login e di accedere ad altre funzionalit\`a.
%TODO
