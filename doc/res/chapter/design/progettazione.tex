\section{Progettazione}

Di seguito vengono presentate e spiegate alcune scelte di design effettuate durante la progettazione del sito.


Come già accennato in altre parti della relazione, si è scelto di utilizzare il Perl e l'HTML in modo tale da ottenere un'altissima modularizzazione: questo ha permesso, durante lo sviluppo del sito, di applicare facilmente modifiche a tutta la struttura HTML, e contemporaneamente ha permesso al sito stesso un'altissima estendibilità. Di JavaScript \`e stato fatto un uso molto limitato, permettendo quindi un'ottima degradazione per tutti gli utenti che non possono eseguire codice JavaScript nel proprio browser. Per quanto riguarda il CSS, inoltre, è stata utilizzata la versione 3.0 per le funzionalità aggiuntive (come effetti sulle transizioni e media query) apprezzabili per gli utenti che utilizzano moderne tecnologie. Per tutti gli altri utenti è stata comunque garantita una degradazione che permettesse una navigazione accettabile nel sito.

\subsubsection{Dispositivi testati}
\begin{itemize}
	\item PC:
		\begin{itemize}
			\item Internet Explorer Versioni: 8, 9, 10, 11;
			\item Firefox Versione: 32.0, 43.0, 44.0;
			\item Google Chrome Versione: 48.0
			\item Opera Versione: 12.01;
			\item Safari Versione: 5.1.7;
		\end{itemize}
	\item Tablet Nexus 7\footnote{Non \`e stato possibile effettuare il downgrade dei browser, in quanto non permesso dallo store}:
		\begin{itemize}
			\item Google Chrome Versione: 48.0;
			\item Firefox Versione Versione: 44;
			\item Opera Versione: 35.0;
		\end{itemize}
	\item Smartphone Lg P880\footnote{Non \`e stato possibile effettuare il downgrade dei browser, in quanto non permesso dallo store}:
		\begin{itemize}
			\item Google Chrome: 48.0;
		\end{itemize}
\end{itemize}
