\section{Progettazione}

Di seguito vengono presentate e spiegate alcune scelte di design effettuate durante la progettazione del sito.


Come già accennato in altre parti della relazione, si è scelto di utilizzare il Perl e l'HTML in modo tale da ottenere un'altissima modularizzazione: questo ha permesso, durante lo sviluppo del sito, di applicare facilmente modifiche a tutta la struttura HTML, e contemporaneamente ha permesso al sito stesso un'altissima estendibilità. Di JavaScript \`e stato fatto un uso molto limitato, permettendo quindi un'ottima degradazione per tutti gli utenti che non possono eseguire codice JavaScript nel proprio browser. Per quanto riguarda il CSS, inoltre, è stata utilizzata la versione 3.0 per le funzionalità aggiuntive (come effetti transizionali e media query) apprezzabili per gli utenti che utilizzano moderne tecnologie. Per tutti gli altri utenti è stata comunque garantita una degradazione che permettesse una navigazione accettabile nel sito.
