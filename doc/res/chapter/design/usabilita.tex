\subsection{Usabilità}
%TODO
\label{form-usabilita}
Come già visto in \ref{form-accessibilita} sono stati adottati specifici accorgimenti per rendere le form e l'inserimento dati accessibili agli utenti. \'E stata posta particolare attenzione inoltre, nel rispettare le regole basilari inerenti usabilit\`a: %elencare quali

\begin{itemize}
    \item \textbf{Le sei W}: L'home page del sito risponde alle seguenti domande:
        \begin{itemize}
            \item \emph{Where}: si riesce a capire direttamente ci\`o che il sito vuole rappresentare, cio\`e una collezione di canzoni e di articoli. In particolare nella prima pagina sono rappresentati gli ultimi inserimenti.
            \item \emph{Who}: l'identit\`a del sito \`e ben rappresentata dal logo in alto a sinistra, che \`e presente in tutte le pagine del sito.
            \item \emph{When}: le due liste presenti in prima pagina rappresentano gli ultimi 5 inserimenti rispettivamente di Articoli e Canzoni in ordine cronologico. All'inizio della lista \`e presente, quindi, l'ultimo elemento inserito, e via via quelli meno recenti in modo tale che l'utente capisca subito quale \`e stato l'ultimo inserimento.
            \item \emph{How}: la barra di navigazione, disposta in alto nella pagina, presenta le sezioni che contengono tutti gli argomenti trattati dal sito.
            \item \emph{What}: l'utente, quando effettua l'accesso al sito, riesce subito a farsi un'idea generale di cosa il sito offre. Grazie, infatti, alle sezioni della barra di navigazione come Artisti o Articoli, e dai titoli presenti come "Ultime Canzoni", si capisce che il sito offre delle informazioni riguardo al mondo della musica.
        \end{itemize}

    \item \textbf{Navbar}: la navbar \`e ben visibile e presente su tutte le pagine. Il colore di una sezione diventa pi\`u chiaro se viene selezionata, in modo da far capire all'utente dove si trova nel sito e viene anche resa non cliccabile per evitare il ricaricamento inutile della pagina. Anche quando viene posizionato il mouse sopra ad una qualsiasi sezione, questa cambia colore, per rappresentare che si tratta di un link.

    \item \textbf{Breadcrumb}: per indicare all'utente la sezione del sito in cui si trova, \`e stato inserito, sotto la barra di navigazione, il percorso che si \`e effettuato dall'home page.

    \item \textbf{Link}: ogni link \`e rappresentato con la sottolineatura, rispettando gli standard, ma \`e stata cambiata la tipica colorazione viola, per rendere i colori del sito pi\`u armoniosi.

    \item \textbf{Foto}: ogni foto \`e nitida e ben visibile, corredata di tag title e alt per dare il massimo delle informazioni possibili.

\end{itemize}
