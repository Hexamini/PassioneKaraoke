\subsection{Gestione dei Dati}
Il sito contiene quattro tipi principali di dati che devono essere gestiti:
\begin{itemize}
    \item \textbf{Utenti}: i quali, oltre ai dati personali, contengono una lista di tutte le preferenze (positive o negative) espresse per le varie canzoni;
    \item \textbf{Articoli}: sono gli articoli che vengono mostrati nell'apposita sezione;
    \item \textbf{Artisti}: contengono al loro interno una lista degli album con le relative canzoni. Ogni canzone \`e corredata da il testo e il video con il relativo karaoke.
    \item \textbf{News}: visualizzate in homepage, contengono le ultime canzoni inserite e gli ultimi articoli
\end{itemize}

\subsubsection{XML}

Per strutturare, e rendere validi i dati nel file XML sono stati creati degli appositi XMLSchema.
Si \`e deciso di utilizzare il modello \textit{``Tende alla Veneziana''} per ogni tipo di dato presente che deve essere gestito.
\`E stata valutata poi l'opportunit\`a di creare anche un elemento globale, presente in tutti i file XMLSchema che gestisca i tag HTML necessari per rendere il contenuto accessibile (ad esempio \textit{\textless span lang=``en''\textgreater}) come tag XML ma, essendo molti, \`e stato deciso di inserire i dati che richiedono questi tag in \textit{\textless![CDATA[testo]]\textgreater} in modo tale da dare la massima possibilit\`a di scelta all'amministratore di inserire qualsiasi tag HTML, non compromettendo la validit/`a totale del file.

\subsection{Trasformate XSLT}

Per visualizzare il contenuto presente nel file XML si \`e deciso di non utilizzare i template XSLT, in quanto la trasformazione da XML a HTML \`e stata resa pi\`u facile dalle librerie Perl.
