\section{Introduzione}

\subsection{Abstract}
L`obiettivo che questo progetto si pone \`e l'implementazione di un sito web dedicato alla consultazione di testi di canzoni (in inglese \textit{lyrics}), e di articoli inerenti al mondo musicale. 
Dalla pagina principale, il visitatore pu\`o accedere alla sezione relativa agli Artisti, a quella relativa agli Articoli o alla pagina del log in.
La sezione Artisti, nello specifico, contiene tutti i cantanti, e per ogni cantante i relativi album. Selezionando poi l'album di interesse, si potr\`a accedere all'elenco delle canzoni in esso contenute e, per ogni canzone, saranno disponibili il testo e il video del karaoke.
Per quando riguarda gli Articoli, questi si potranno consultare nella relativa sezione, dove compariranno in ordine cronologico a partire dal pi\`u recente.
Infine effettuando il log in, l'Utente registrato accede alla sua pagina personale in cui ritrova le sue canzoni preferite. L'Utente registrato ha infatti la possibilit\`a di esprimere un giudizio positivo o negativo per ogni canzone; e le canzoni per cui esprime un giudizio positivo vengono aggiunte direttamente nella pagina personale, dove possono essere comodamente consultate dall'Utente anche in momenti successivi.
L'Amministratore è un tipo particolare di Utente che pu\`o ampliare il catalogo aggiungendo nuovi Artisti, con i relativi album e canzoni, oppure nuovi album o nuove canzoni in Artisti e Album gi\`a esistenti. Può inoltre aggiornare gli utenti sulle novità del mondo della musica tramite la creazione di news (articoli).


\newpage %interruzione di pagina


\subsection{A chi è rivolto il sito}
Il sito sar\`a consultato da utenti molto vari, in quanto non tratta una materia di nicchia ma \`e inerente a un mondo di interesse pressoch\`e universale com'\`e quello della musica. Questo grande bacino di utenti comporta, di conseguenza, la necessit\`a di porre particolare attenzione all'accessibilit\`a generale nella fase di implementazione. Proprio con l'obiettivo di migliorare l`accessibilit\`a, per esempio, i contenuti saranno inseriti da un \textbf{Admin}, che si occuperà di scrivere articoli in formato \textit{HTML}, rispettandone tutte le convenzioni e inserendo gli opportuni tag per rendere la lettura dei contenuti fruibile anche agli utenti che hanno problemi visivi.


\subsection{Credenziali d'accesso al sito}
Per accedere al sito sono state creade delle credenziali di default.
\begin{itemize}

    \item Livello admin:
    \begin{itemize}

        \item Utente: \textit{admin}
        \item Password: \textit{password}

    \end{itemize}

    \item Livello utente:
    \begin{itemize}

        \item Utente: \textit{utente}
        \item Password: \textit{utente}

    \end{itemize}

\end{itemize}
